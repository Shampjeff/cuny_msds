\PassOptionsToPackage{unicode=true}{hyperref} % options for packages loaded elsewhere
\PassOptionsToPackage{hyphens}{url}
%
\documentclass[]{article}
\usepackage{lmodern}
\usepackage{amssymb,amsmath}
\usepackage{ifxetex,ifluatex}
\usepackage{fixltx2e} % provides \textsubscript
\ifnum 0\ifxetex 1\fi\ifluatex 1\fi=0 % if pdftex
  \usepackage[T1]{fontenc}
  \usepackage[utf8]{inputenc}
  \usepackage{textcomp} % provides euro and other symbols
\else % if luatex or xelatex
  \usepackage{unicode-math}
  \defaultfontfeatures{Ligatures=TeX,Scale=MatchLowercase}
\fi
% use upquote if available, for straight quotes in verbatim environments
\IfFileExists{upquote.sty}{\usepackage{upquote}}{}
% use microtype if available
\IfFileExists{microtype.sty}{%
\usepackage[]{microtype}
\UseMicrotypeSet[protrusion]{basicmath} % disable protrusion for tt fonts
}{}
\IfFileExists{parskip.sty}{%
\usepackage{parskip}
}{% else
\setlength{\parindent}{0pt}
\setlength{\parskip}{6pt plus 2pt minus 1pt}
}
\usepackage{hyperref}
\hypersetup{
            pdftitle={Chapter 3 - Probability},
            pdfborder={0 0 0},
            breaklinks=true}
\urlstyle{same}  % don't use monospace font for urls
\usepackage[margin=1in]{geometry}
\usepackage{color}
\usepackage{fancyvrb}
\newcommand{\VerbBar}{|}
\newcommand{\VERB}{\Verb[commandchars=\\\{\}]}
\DefineVerbatimEnvironment{Highlighting}{Verbatim}{commandchars=\\\{\}}
% Add ',fontsize=\small' for more characters per line
\usepackage{framed}
\definecolor{shadecolor}{RGB}{248,248,248}
\newenvironment{Shaded}{\begin{snugshade}}{\end{snugshade}}
\newcommand{\AlertTok}[1]{\textcolor[rgb]{0.94,0.16,0.16}{#1}}
\newcommand{\AnnotationTok}[1]{\textcolor[rgb]{0.56,0.35,0.01}{\textbf{\textit{#1}}}}
\newcommand{\AttributeTok}[1]{\textcolor[rgb]{0.77,0.63,0.00}{#1}}
\newcommand{\BaseNTok}[1]{\textcolor[rgb]{0.00,0.00,0.81}{#1}}
\newcommand{\BuiltInTok}[1]{#1}
\newcommand{\CharTok}[1]{\textcolor[rgb]{0.31,0.60,0.02}{#1}}
\newcommand{\CommentTok}[1]{\textcolor[rgb]{0.56,0.35,0.01}{\textit{#1}}}
\newcommand{\CommentVarTok}[1]{\textcolor[rgb]{0.56,0.35,0.01}{\textbf{\textit{#1}}}}
\newcommand{\ConstantTok}[1]{\textcolor[rgb]{0.00,0.00,0.00}{#1}}
\newcommand{\ControlFlowTok}[1]{\textcolor[rgb]{0.13,0.29,0.53}{\textbf{#1}}}
\newcommand{\DataTypeTok}[1]{\textcolor[rgb]{0.13,0.29,0.53}{#1}}
\newcommand{\DecValTok}[1]{\textcolor[rgb]{0.00,0.00,0.81}{#1}}
\newcommand{\DocumentationTok}[1]{\textcolor[rgb]{0.56,0.35,0.01}{\textbf{\textit{#1}}}}
\newcommand{\ErrorTok}[1]{\textcolor[rgb]{0.64,0.00,0.00}{\textbf{#1}}}
\newcommand{\ExtensionTok}[1]{#1}
\newcommand{\FloatTok}[1]{\textcolor[rgb]{0.00,0.00,0.81}{#1}}
\newcommand{\FunctionTok}[1]{\textcolor[rgb]{0.00,0.00,0.00}{#1}}
\newcommand{\ImportTok}[1]{#1}
\newcommand{\InformationTok}[1]{\textcolor[rgb]{0.56,0.35,0.01}{\textbf{\textit{#1}}}}
\newcommand{\KeywordTok}[1]{\textcolor[rgb]{0.13,0.29,0.53}{\textbf{#1}}}
\newcommand{\NormalTok}[1]{#1}
\newcommand{\OperatorTok}[1]{\textcolor[rgb]{0.81,0.36,0.00}{\textbf{#1}}}
\newcommand{\OtherTok}[1]{\textcolor[rgb]{0.56,0.35,0.01}{#1}}
\newcommand{\PreprocessorTok}[1]{\textcolor[rgb]{0.56,0.35,0.01}{\textit{#1}}}
\newcommand{\RegionMarkerTok}[1]{#1}
\newcommand{\SpecialCharTok}[1]{\textcolor[rgb]{0.00,0.00,0.00}{#1}}
\newcommand{\SpecialStringTok}[1]{\textcolor[rgb]{0.31,0.60,0.02}{#1}}
\newcommand{\StringTok}[1]{\textcolor[rgb]{0.31,0.60,0.02}{#1}}
\newcommand{\VariableTok}[1]{\textcolor[rgb]{0.00,0.00,0.00}{#1}}
\newcommand{\VerbatimStringTok}[1]{\textcolor[rgb]{0.31,0.60,0.02}{#1}}
\newcommand{\WarningTok}[1]{\textcolor[rgb]{0.56,0.35,0.01}{\textbf{\textit{#1}}}}
\usepackage{graphicx,grffile}
\makeatletter
\def\maxwidth{\ifdim\Gin@nat@width>\linewidth\linewidth\else\Gin@nat@width\fi}
\def\maxheight{\ifdim\Gin@nat@height>\textheight\textheight\else\Gin@nat@height\fi}
\makeatother
% Scale images if necessary, so that they will not overflow the page
% margins by default, and it is still possible to overwrite the defaults
% using explicit options in \includegraphics[width, height, ...]{}
\setkeys{Gin}{width=\maxwidth,height=\maxheight,keepaspectratio}
\setlength{\emergencystretch}{3em}  % prevent overfull lines
\providecommand{\tightlist}{%
  \setlength{\itemsep}{0pt}\setlength{\parskip}{0pt}}
\setcounter{secnumdepth}{0}
% Redefines (sub)paragraphs to behave more like sections
\ifx\paragraph\undefined\else
\let\oldparagraph\paragraph
\renewcommand{\paragraph}[1]{\oldparagraph{#1}\mbox{}}
\fi
\ifx\subparagraph\undefined\else
\let\oldsubparagraph\subparagraph
\renewcommand{\subparagraph}[1]{\oldsubparagraph{#1}\mbox{}}
\fi

% set default figure placement to htbp
\makeatletter
\def\fps@figure{htbp}
\makeatother

\usepackage{geometry}
\usepackage{multicol}
\usepackage{multirow}

\title{Chapter 3 - Probability}
\author{}
\date{\vspace{-2.5em}}

\begin{document}
\maketitle

\textbf{Dice rolls.} (3.6, p.~92) If you roll a pair of fair dice, what
is the probability of

\begin{enumerate}
\def\labelenumi{(\alph{enumi})}
\tightlist
\item
  getting a sum of 1?
\end{enumerate}

Zero. The lowest possible sum is 2.

\begin{enumerate}
\def\labelenumi{(\alph{enumi})}
\setcounter{enumi}{1}
\tightlist
\item
  getting a sum of 5?
\end{enumerate}

The first die can be 1,2,3 or 4. The second can only one number based on
the first \(\frac{4}{6}*\frac{1}{6}=\frac{1}{9}\) or 11.11\%

\begin{enumerate}
\def\labelenumi{(\alph{enumi})}
\setcounter{enumi}{2}
\tightlist
\item
  getting a sum of 12?
\end{enumerate}

There is only one way to get a sum of 12, double 6s. \(\frac{1}{36}\) or
2.77\%

\begin{center}\rule{0.5\linewidth}{0.5pt}\end{center}

\clearpage

\textbf{Poverty and language}. (3.8, p.~93) The American Community
Survey is an ongoing survey that provides data every year to give
communities the current information they need to plan investments and
services. The 2010 American Community Survey estimates that 14.6\% of
Americans live below the poverty line, 20.7\% speak a language other
than English (foreign language) at home, and 4.2\% fall into both
categories.

\begin{enumerate}
\def\labelenumi{(\alph{enumi})}
\tightlist
\item
  Are living below the poverty line and speaking a foreign language at
  home disjoint?
\end{enumerate}

No they are not disjoint. You can be both a non-native speaker and below
the poverty line.

\begin{enumerate}
\def\labelenumi{(\alph{enumi})}
\setcounter{enumi}{1}
\tightlist
\item
  Draw a Venn diagram summarizing the variables and their associated
  probabilities.
\end{enumerate}

\begin{Shaded}
\begin{Highlighting}[]
\NormalTok{ ---------------- - - - - - - - }
\NormalTok{|poverty  |both  | non-english |}
\NormalTok{|_14%_____|_4%___|___20%_______|}
\end{Highlighting}
\end{Shaded}

\begin{enumerate}
\def\labelenumi{(\alph{enumi})}
\setcounter{enumi}{2}
\tightlist
\item
  What percent of Americans live below the poverty line and only speak
  English at home?
\end{enumerate}

P(pvoerty and non-english) = 14.6\% x 79.3\% = 11.57\%

\begin{enumerate}
\def\labelenumi{(\alph{enumi})}
\setcounter{enumi}{3}
\tightlist
\item
  What percent of Americans live below the poverty line or speak a
  foreign language at home?
\end{enumerate}

the stats definition of OR is inclusive so this would be poverty or
`flh' or both 14.2\% + 20.7\% + 4.2\% = 39.1\%

\begin{enumerate}
\def\labelenumi{(\alph{enumi})}
\setcounter{enumi}{4}
\tightlist
\item
  What percent of Americans live above the poverty line and only speak
  English at home?
\end{enumerate}

85.4\% * 79.3\% = 67.8\%

\begin{enumerate}
\def\labelenumi{(\alph{enumi})}
\setcounter{enumi}{5}
\tightlist
\item
  Is the event that someone lives below the poverty line independent of
  the event that the person speaks a foreign language at home?
\end{enumerate}

No.~If you know something about either percentage (below poverty of
speaking non-english at home) you can determine information about the
other event.

\begin{center}\rule{0.5\linewidth}{0.5pt}\end{center}

\clearpage

\textbf{Assortative mating}. (3.18, p.~111) Assortative mating is a
nonrandom mating pattern where individuals with similar genotypes and/or
phenotypes mate with one another more frequently than what would be
expected under a random mating pattern. Researchers studying this topic
collected data on eye colors of 204 Scandinavian men and their female
partners. The table below summarizes the results. For simplicity, we
only include heterosexual relationships in this exercise.

\(\begin{center} \begin{tabular}{ll ccc c}  & & \multicolumn{3}{c}{\textit{Partner (female)}} \\ \cline{3-5}  & & Blue & Brown & Green & Total \\ \cline{2-6}  & Blue & 78 & 23 & 13 & 114 \\ \multirow{2}{*}{\textit{Self (male)}} & Brown & 19 & 23 & 12 & 54 \\  & Green & 11 & 9 & 16 & 36 \\ \cline{2-6}  & Total & 108 & 55 & 41 & 204 \end{tabular} \end{center}\)

\begin{enumerate}
\def\labelenumi{(\alph{enumi})}
\tightlist
\item
  What is the probability that a randomly chosen male respondent or his
  partner has blue eyes?
\end{enumerate}

There are 114 males with blue eyes and 108 females with blue eyes. This
captures all couples with blues eye somewhere, but we need to subrtract
the double counted blue/blue couples. \(114+108-78 = 144\) and
\(144/204 = 70.5%\) have blue eyes in either of the partners.

\begin{enumerate}
\def\labelenumi{(\alph{enumi})}
\setcounter{enumi}{1}
\tightlist
\item
  What is the probability that a randomly chosen male respondent with
  blue eyes has a partner with blue eyes?
\end{enumerate}

This is conditional probability, what is the probability of a male=blue
given female=blue

\(P(f=blue|m=blue) = \frac{\frac{78}{204} }{\frac{114}{204}} = 69.4%\)

\begin{enumerate}
\def\labelenumi{(\alph{enumi})}
\setcounter{enumi}{2}
\tightlist
\item
  What is the probability that a randomly chosen male respondent with
  brown eyes has a partner with blue eyes? What about the probability of
  a randomly chosen male respondent with green eyes having a partner
  with blue eyes?
\end{enumerate}

\(P(f=blue|m=brown) = \frac{\frac{19}{204} }{\frac{54}{204}} = 35.2%\)

\(P(f=blue|m=green) = \frac{\frac{11}{204} }{\frac{36}{204}} = 30.5%\)

\begin{enumerate}
\def\labelenumi{(\alph{enumi})}
\setcounter{enumi}{3}
\tightlist
\item
  Does it appear that the eye colors of male respondents and their
  partners are independent? Explain your reasoning.
\end{enumerate}

They seem dependent. You can determine information about possible eye
color of one partner from the other.

\begin{center}\rule{0.5\linewidth}{0.5pt}\end{center}

\clearpage

\textbf{Books on a bookshelf}. (3.26, p.~114) The table below shows the
distribution of books on a bookcase based on whether they are nonfiction
or fiction and hardcover or paperback.

\(\begin{center} \begin{tabular}{ll cc c}  & & \multicolumn{2}{c}{\textit{Format}} \\ \cline{3-4}  & & Hardcover & Paperback & Total \\ \cline{2-5} \multirow{2}{*}{\textit{Type}} & Fiction & 13 & 59 & 72 \\  & Nonfiction& 15 & 8 & 23 \\ \cline{2-5}  & Total & 28 & 67 & 95 \\ \cline{2-5} \end{tabular} \end{center}\)

\begin{enumerate}
\def\labelenumi{(\alph{enumi})}
\tightlist
\item
  Find the probability of drawing a hardcover book first then a
  paperback fiction book second when drawing without replacement.
\end{enumerate}

28/95 * 59/94 = 18.5\%

\begin{enumerate}
\def\labelenumi{(\alph{enumi})}
\setcounter{enumi}{1}
\tightlist
\item
  Determine the probability of drawing a fiction book first and then a
  hardcover book second, when drawing without replacement.
\end{enumerate}

72/95 * 27/94 = 21.7\%

\begin{enumerate}
\def\labelenumi{(\alph{enumi})}
\setcounter{enumi}{2}
\tightlist
\item
  Calculate the probability of the scenario in part (b), except this
  time complete the calculations under the scenario where the first book
  is placed back on the bookcase before randomly drawing the second
  book.
\end{enumerate}

72/95 * 28/95 = 22.3\%

\begin{enumerate}
\def\labelenumi{(\alph{enumi})}
\setcounter{enumi}{3}
\tightlist
\item
  The final answers to parts (b) and (c) are very similar. Explain why
  this is the case.
\end{enumerate}

Because the same book is both ficition and hardcover so both the
numerator and denominator decreased by the same amount in part (b). Also
one book out of 95 isn't going be make much change.

\begin{center}\rule{0.5\linewidth}{0.5pt}\end{center}

\clearpage

\textbf{Baggage fees}. (3.34, p.~124) An airline charges the following
baggage fees: \$25 for the first bag and \$35 for the second. Suppose
54\% of passengers have no checked luggage, 34\% have one piece of
checked luggage and 12\% have two pieces. We suppose a negligible
portion of people check more than two bags.

\begin{enumerate}
\def\labelenumi{(\alph{enumi})}
\tightlist
\item
  Build a probability model, compute the average revenue per passenger,
  and compute the corresponding standard deviation.
\end{enumerate}

\(E(X) = 0*(.54) + 25*(.34) + (35+25)*(.12) = 15.7\) dollars per
customer is expected.

\(\sigma^{2} = (x_{i} - \mu)^{2}*P(x_{i}) =(-15.7)^{2} + (25-15.7)^{2}*.34 + (60-15.7)^{2}*(.12) = 398.01\),

\(\sigma = 19.95\)

\begin{enumerate}
\def\labelenumi{(\alph{enumi})}
\setcounter{enumi}{1}
\tightlist
\item
  About how much revenue should the airline expect for a flight of 120
  passengers? With what standard deviation? Note any assumptions you
  make and if you think they are justified.
\end{enumerate}

\(\sigma_{120} = \sqrt(120*Var) = \sqrt(120*398.01) = 218.54\)

We except to make \(\$15.7*120 = \$1884 \pm 218.54\)

\begin{center}\rule{0.5\linewidth}{0.5pt}\end{center}

\clearpage

\textbf{Income and gender}. (3.38, p.~128) The relative frequency table
below displays the distribution of annual total personal income (in 2009
inflation-adjusted dollars) for a representative sample of 96,420,486
Americans. These data come from the American Community Survey for
2005-2009. This sample is comprised of 59\% males and 41\% females.

\(\begin{center} \begin{tabular}{lr}  \hline \textit{Income} & \textit{Total} \\  \hline \$1 to \$9,999 or loss & 2.2\% \\ \$10,000 to \$14,999 & 4.7\% \\ \$15,000 to \$24,999 & 15.8\% \\ \$25,000 to \$34,999 & 18.3\% \\ \$35,000 to \$49,999 & 21.2\% \\ \$50,000 to \$64,999 & 13.9\% \\ \$65,000 to \$74,999 & 5.8\% \\ \$75,000 to \$99,999 & 8.4\% \\ \$100,000 or more & 9.7\% \\  \hline \end{tabular} \end{center}\)

\begin{enumerate}
\def\labelenumi{(\alph{enumi})}
\tightlist
\item
  Describe the distribution of total personal income.
\end{enumerate}

This appears to be relatively normal (Gaussian). It is mostly symetric
and does not appear to be noticably skewed.

\begin{enumerate}
\def\labelenumi{(\alph{enumi})}
\setcounter{enumi}{1}
\tightlist
\item
  What is the probability that a randomly chosen US resident makes less
  than \$50,000 per year?
\end{enumerate}

This would be 62.2\%.

\begin{enumerate}
\def\labelenumi{(\alph{enumi})}
\setcounter{enumi}{2}
\tightlist
\item
  What is the probability that a randomly chosen US resident makes less
  than \$50,000 per year and is female? Note any assumptions you make.
\end{enumerate}

Assuming this survey discriminates against gender non-conforming people
such that there are only two genders represented. (.41)*(.622) = 25.5\%

\begin{enumerate}
\def\labelenumi{(\alph{enumi})}
\setcounter{enumi}{3}
\tightlist
\item
  The same data source indicates that 71.8\% of females make less than
  \$50,000 per year. Use this value to determine whether or not the
  assumption you made in part (c) is valid.
\end{enumerate}

This is a huge change. Something has to be wrong, either there is
another group of people who were lumped into the female category and who
make more money or there is something wrong with how income is being
reported or collected.

\end{document}
